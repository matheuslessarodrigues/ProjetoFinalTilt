2.2.1 Incubadoras
2.2.2 Aceleradoras
2.2.4 Mentores
2.2.5 Crowd Funding
2.2.3 Investidores Anjo

Incubadoras
----

O que é uma incubadora

Uma incubadora de empresas é uma instuição que tem por objetivo auxiliar a criação e o desenvolvimento de startups acompanhando-as em seus primeiros anos de vida. Elas dispõem de um espaço físico onde suas startups podem se estabelecer, o que muitas vezes é vital para o desenvolvimento de seus respectivos produtos. Acaba tornando-se inevitável a troca de experiências uma vez que as empresas incubadas são vizinhas ou simplesmente compartilham o mesmo escritório com outras empresas. Uma incubadora pode ser sumarizada como um centro de troca de experiência e inovação para startups. As empresas costuma continuar incubadas durante alguns anos até atingirem um crescimento crítico que as force a procurar um novo escritório.

Além do espaço físico para trabalho, incubadoras possuem contatos de experientes empreendedores e investidores que oferecem mentorias às startups com a finalidade de ajudá-las em suas estratégias de negócio. Esses mentores podem ser ex-membros da incubadora repassando seus conhecimentos adquiridos adiante, empreendedores em ascensão ou veteranos da indústria que não seriam comunicáveis não fosse a incubadora.

Dados os benefícios trazidos por uma incubadora, as vagas para seus escritórios são altamente concorridas. Como consequência, incubadoras têm de selecionar devidamente as empresas que farão parte de seu "portfolio". Por conseguinte, investidores exergam as empresas incubadas como um grupo seleto e com potencial o que, inerentemente, aumenta as chances dessas empresas conseguirem levantar fundos para seus projetos. Dessa forma, a incubadora funciona como um "selo de garantia" para eles.

* Listar algum exemplo brasileiro.

[1]
http://www.forbes.com/sites/georgedeeb/2014/08/28/is-a-startup-incubator-or-accelerator-right-for-you/

[2]
http://techli.com/2012/06/how-incubators-hurt-startups/#.

[3]
http://nabeelqadeer.com/the-need-for-startup-incubators/
