2.2.1 Incubadoras
2.2.2 Aceleradoras
2.2.4 Mentores
2.2.5 Crowd Funding
2.2.3 Investidores Anjo

Incubadoras
----

Uma incubadora de empresas é uma instuição que tem por objetivo auxiliar a criação e o desenvolvimento de startups acompanhando-as em seus primeiros anos de vida. Elas dispõem de um espaço físico onde suas startups podem se estabelecer, o que muitas vezes é vital para o desenvolvimento de seus respectivos produtos. Acaba tornando-se inevitável a troca de experiências uma vez que as empresas incubadas são vizinhas ou simplesmente compartilham o mesmo escritório com outras empresas. Uma incubadora pode ser sumarizada como um centro de troca de experiência e inovação para startups. As empresas costuma continuar incubadas durante alguns anos até atingirem um crescimento crítico que as force a procurar um novo escritório.

Além do espaço físico para trabalho, incubadoras possuem contatos de experientes empreendedores e investidores que oferecem mentorias às startups com a finalidade de ajudá-las em suas estratégias de negócio. Esses mentores podem ser ex-membros da incubadora repassando seus conhecimentos adquiridos adiante, empreendedores em ascensão ou veteranos da indústria que não seriam comunicáveis não fosse a incubadora.

Dados os benefícios trazidos por uma incubadora, as vagas para seus escritórios são altamente concorridas. Como consequência, incubadoras têm de selecionar devidamente as empresas que farão parte de seu "portfolio". Por conseguinte, investidores exergam as empresas incubadas como um grupo seleto e com potencial o que, inerentemente, aumenta as chances dessas empresas conseguirem levantar fundos para seus projetos. Dessa forma, a incubadora funciona como um "selo de garantia" para eles.

* Listar algum exemplo brasileiro.

[1]
http://www.forbes.com/sites/georgedeeb/2014/08/28/is-a-startup-incubator-or-accelerator-right-for-you/

[2]
http://techli.com/2012/06/how-incubators-hurt-startups/#.

[3]
http://nabeelqadeer.com/the-need-for-startup-incubators/


Aceleradoras
----

Assim como uma incubadora, aceleradoras têm por objetivo auxiliar no desenvolvimento de startups. Porém, possuem uma relação mais curta e ágil com as startups. Ao invés de disponibilizarem escritórios, as aceleradoras normalmente irão ajudar com um pequeno investimento e uma série intensa de mentorias, eventos e treinamentos durante um período que costuma variar entre 3 e 8 meses. Assim como as incubadoras, seus programas de seleção aceitam mais de uma empresa por vez, o que permite que durante esse período de aceleração, as startups possam trocar experiências e se ajudar.

Próximo ao fim do período de aceleração, é comum a aceleradora organizar uma série de "Demo Days" para que as startups possam apresentar seus projetos para potenciais investidores. "Demo Days", ou dias de demonstração, são eventos próprios para investidores conhecerem novos projetos sendo criados por startups. Para se destacarem nesses eventos, as startups precisam treinar e trabalhar bastante seu "pitch" que é uma técnica de venda em que o apresentador tem um curto tempo (aproximadamente 3 minutos) para apresentar e descrever seu produto ou projeto e tentar vendê-lo para sua platéia. Para isso, as startups devem ser consisas, sucintas e bem claras quando apresentarem seu projeto.

Novamente, como aceleradoras são altamente concorridas, seus processos seletivos são bem criteriosos. Logo, startups precisam ter um preparo e planejamento antes de tentarem a inscrição em uma aceleradora. Dessa forma, essas empresas melhor preparadas irão tirar maior proveito do programa que, por consequência, terão maior chances de levantar investimento que empresas que não foram aceleradas.

* listar mais exemplos brasileiros


"Crowd Funding"
----

"Crowd Funding", ou financiamento coletivo, é uma forma moderna de captação de recursos que vem ganhando força e atenção cada vez mais. Como uma forma de levantar fundos, ele tem por objetivo o financiamento de um projeto, porém o que o diferencia de métodos de investimento clássico é que o dinheiro não vem de um pequeno grupo de investidores em que cada um aplica uma quantia considerável, mas sim de uma grande quantidade de "micro investidores". A idéia é atrair um grande número de pessoas cativando-as com sua proposta de produto ou projeto. Assim, é possível que cada uma delas invista quantias razoáveis, geralmente entre $5 e $500, e o projeto é financiado pela quantidade.

O interessante do "Crowd Funding" é que não são necessários contratos complexos para o investimento, até porque ele tem como princípio atingir pessoas que provavelmente não têm costume de aplicação de dinheiro. Por causa de sua natureza simples, para os "micro investidores", é como se estivessem comprando algo pois, para cada quantia investida, eles ganham algum presente de volta como adesivos, camisa, primeiras versões do produto, entre outros. Enquanto que para a equipe idealizando o projeto, é ótimo pois não precisam lidar com investidores de verdade ou vender parte de sua startup e podem continuar com o foco no projeto proposto.

Como consequência, logo surgiram plataformas online de "Crowd Funding" tais como o "Kickstarter" [4] uma das mais famosas. Com essa facilidade, é comum encontrar estúdios independentes cadastrarem seu novo jogo em uma dessas plataformas. Dessa forma, a startup recebe uma página na web onde ela descreve o projeto e atualiza o conteúdo conforma seu desenvolvimento, define as diferentes quantias de investimento bem como a recompensa para cada um, e estabelecem o objetivo de investimento: qual o total mínimo que esperam receber.

Por ser uma ferramenta online, é mais fácil prospectar potenciais "micro investidores". O resultado disso é que geralmente é construída uma comunidade ao redor do projeto durante o tempo limitado de campanha. Essas pessoas são aquelas que mais acreditaram na visão da startup e que enxergam grande valor em seu projeto, elas são chamadas de "early adopters". Por causa disso, é comum ver uma campanha de "Crowd Funding" sendo realizada também como forma de validar que pessoas se interessarão pelo produto.

Exemplo de plataforma de "Crowd Funding" brasileira: Catarse [5]

[4] Kickstarter
https://www.kickstarter.com/

[5] Catarse
http://www.catarse.me/


Investidor Anjo
----

Investidores anjo são indivíduos que provêem investimento para pequenas startups. Em troca, eles podem se tornar donos de parte da startup ou ter vantagem na venda de partes da startup. [6] Esses investidores autônomos muitas vezes são ex-empreendedores que desejam continuar acompanhando de perto inovações da área em que atuava ou simplesmente pessoas ricas que conhecem de perto o trabalho da startup alvo quando o objetivo de investir vai além de apenas retorno financeiro.

Investidores anjo trabalham com investimento de alto risco por se tratar de startups e de não serem uma agência especializada. Entretanto, é comum ver anjos trabalharem em conjunto formando "angel groups" (grupos de anjos) em que é possível trocarem experiências, terem segundas opiniões quanto a uma startup bem como multiplicar seu poder de investimento também. Por outro lado, como geralmente são responsáveis pelo primeiro levantamento de fundos de uma startup, eles estarão sussetíveis ao fenômeno de diluição ("dilution") [7] e, portanto, cobram um grande retorno de investimento.

Por causa de sua natureza informal e de alto risco, os anjos devem estar sempre à procura de novas empresas para investir. Dessa forma é necessário estar continuamente expandindo sua rede de contatos. Inclusive, existem hoje plataformas para isso como a "Angel List" [8]. Empreendedores cadastram sua startup e investidores anjo podem navegar pelas empresas e projetos que julgarem mais interessantes.


[6] Angel Investor
http://www.investopedia.com/terms/a/angelinvestor.asp

[7] Stock Dilution
https://en.wikipedia.org/wiki/Stock_dilution

[8] Angel List
https://angel.co/
