Sete Estranhos Fazendo Jogo Numa Cozinha
----

Fim de 2012, a "Critical Studio", um estúdio de médio porte localizado no centro da cidade do Rio de Janeiro, decidiu que queria criar oportunidades para fomentar a indústria local de video-games. Já havia alguns anos que começaram o desenvolvimento de seu maior jogo: "Dungeonland". Dada a experiência adquirida durante esse período, era claro que trabalhar e se sustentar criando video-games no Brasil não é tarefa fácil, uma vez que não existem muitos incentivos por parte do governo como também é raro encontrar talentos para trabalhar na área.

Começando como um experimento, membros da "Critical" chamaram ao seu escritório sete dos que julgaram os melhores currículos recebidos ao longo do tempo. Em uma tarde, estavam reunidos sete estranhos no que era a cozinha do estúdio e a eles fora dito: "vocês podem usar esses espaço (a sala de teste/cozinha) à vontade. O que vocês fizerem aqui será propriedade de vocês e somente de vocês. Nós trabalhamos na outra sala, fiquem à vontade para vir falar conosco a qualquer momento". Alguns meses se passaram e a equipe da cozinha se alterou um pouco. Alguns sairam, outros entraram e finalmente o grupo se emancipou e formou o que seria o mais novo estúdio de desenvolvimento de jogos do Rio de Janeiro: a "BitCake Studio".


Project Tilt
----

Como uma desenvolvedora de jogos eletrônicos, a "BitCake" começava o desenvolvimento de seu primeiro projeto: o "Project Tilt". O jogo começou como um protótipo de "local multiplayer" porém, já possuía a visão megalomaníaca de ser um jogo online, em tempo real em que centenas de milhares de pessoas jogariam por dia. Dessa forma, eram necessários foco e planejamento para efetivamente alcançar o objetivo.

Desde aquela época, era claro que um jogo que depende de outras pessoas para ser divertido precisaria de muitos jogadores online ao mesmo tempo. Dessa forma, o Tilt era concebido como um jogo gratuito a se jogar de modo a minimizar as barreiras de entrada uma vez que não seria necessário comprar o jogo para então começar a jogar. Além disso, ele seria jogado diretamente do navegador para que não houvesse a necessidade de baixar qualquer executável. Esse é um dos pilares do jogo: ele vai direto ao ponto, basta acessar o web site e o jogador já estava pronto para se divertir com seus amigos online.

Desenvolver um jogo requer paciência e iteração. Raramente acertamos de primeira, fazendo com que os desenvolvedores repitam inúmeras vezes o processo de fazer pequenas alterações, testar, avaliar, repetir. Por acaso, a equipe da "Critical" era muito boa nisso e estavam dispostos a ajudar o "Tilt" neste aspecto. Como consequência, muito antes de ter uma arte agradável com "cara de jogo", ele já tinha um "gameplay" sólido que o deixava gostoso de se jogar.

Entretanto, não é apenas jogabilidade que se testa apesar de sua grande importância. Quando as regras do jogo ainda estavam sendo formadas, por exemplo: há a necessidade de recarregar a arma quando acaba a munição?; Quantos jogadores se enfrentam ao mesmo tempo?; Como é o esquema de armamentos?; etc, é que são necessários testes para moldar como será o jogo propriamente. Naturalmente, os primeiros testes eram entre a própria equipe da "BitCake" com eventuais ajudas de membros da "Critical" quando assim podiam. Porém não é sadável manter uma equipe reduzida de teste pois as avaliações podem começar a se tornar enviezadas portanto, assim que possível, foi criado um grupo no Facebook de testadores do "Tilt" para que conhecidos e afins pudessem testar o jogo junto com a equipe. A medida foi natural uma vez que o jogo já se encontra na web e podia ser jogado diretamente pelo Facebook o que tirava muita fricção do caminho dos testadores. Além disso, podíamos finalmente testar a experiência completa do jogo através uma partida com dez jogadores simultâneos.
