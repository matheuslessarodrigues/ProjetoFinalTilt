2.3 Ganhando dinheiro com o jogo Free to play

2.3.1 - Diferença entre Free to play e premium
A principal diferença entre um jogo free-to-play e um jogo premium é que no caso do premium o jogador gasta dinheiro antes mesmo de começar a jogar e todo jogador gasta um valor fixo, dez reais, vinte reais, cem reais, dependendo do jogo, enquanto do no free-to-play o jogador gasta dinheiro para comprar coisas a mais no jogo, por exemplo, algumas areas ou alguns itens só podem ser adquiridos com dinheiro ou com muito tempo de jogo, sendo assim as pessoas que não querem investir muito tempo no jogo, acabam investindo dinheiro. É importante entender também que não são todos os jogadores que pagam por esses itens extras e os que pagam demoram um certo tempo para começar a pagar, ou seja, se o jogador jogou seu jogo hoje e nunca mais abriu o jogo, a chance dele gastar algo é praticamente nula, então para entender se seu jogo será um sucesso ou não, é necessario primeiro entender algumas métricas.

2.3.2 - Metricas

2.3.2.1 - Envolvimento
2.3.2.1.1 - Sessions/DAU
Essa métrica refere-se a quantas vezes um jogador abre o jogo no mesmo dia em média. É claro que esse numero depende do seu jogo, se uma sessão do seu jogo costuma durar várias horas, como é o caso dos RPG's, seu jogo tende a ter menos sessões, por outro lado jogos que ocupam menos tempo, jogos casuais no geral, tendem a ter mais sessões. Um numero razoável para essa métrica é 3(três), porém jogos que consomem menos tempo pode facilmente atingir 4 ou 5.

2.3.2.1.2 - DAU/MAU
A razão DAU/MAU representa o quão "viciante", ela representa a porcentagem de usuários que entrou no jogo hoje e também entrou nesse mês. Essa métrica é importante para saber a porcentagem de usuarios que retornaram ao jogo também. Um numero bom para essa métrica é por volta de 0.2.

2.3.2.2 - Retenção - OMG ESSA VADIA!
Em primeiro lugar retenção serve para medir o quanto você está retendo o seu jogador. Hoje em dia, existem duas maneiras de medir a retenção de um jogo.

A primeira delas, mais conhecida, e também mais importante para os investidores é, supondo que X numero de jogadores instalaram seu jogo no dia 0, quantos desses X jogadores voltaram no dia 1? E no dia 7? E no dia 30? Existem várias ferramentas de analytics que já vem com essa implementação e são bem fáceis de integrar.
Uma boa métrica de retenção no caso de Mobile é 40-20-10, 40% para dia 1, 20% para dia 7, e 10% para dia 30, normalmente são os primeiros numeros que atraem a atenção de investidores e/ou publicadoras.
É importante lembrar que essa métrica varia de plataforma para plataforma, ou seja o mesmo jogo terá retenções diferentes em plataformas diferentes (Web, Mobile, Consoles), e também varia por genero do jogo, por exemplo um jogo passa-fase "morre" quando o jogador completa todas as fases, logo este terá uma retenção menor do que um jogo competitivo pois enquanto houver jogadores competindo entre si, o jogo não morrerá.

A segunda Maneira de medir retenção... (pesquisar mais...)


2.3.2.3 - Monetização

2.3.2.3.1 - ARPDAU
Average Revenue Per Daily Active User (ARPDAU) significa quanto um usuário gera em média de receita por dia, é uma das métricas mais comuns no mundo dos jogos movéis pois com ela o desenvolvedor tem como saber como seu jogo está caminhando diariamente. E se o desenvolvedor souber seu gasto diario por usuário e esse ARPDAU for maior do que o gasto diário por usuário, ele sabe que pode investir em aquisição de mais usuários pois ele vai aumentar sua receita. Um primeiro passo é atingir Us$ 0.05 de ARPDAU, porém um jogo com uma monetização boa costuma atingir de U$0.15 à U$ 0.25.

2.3.2.3.2 - ARPU
Average Revenue Per User (ARPU) significa média de receita por usuário, esta métrica mede quanto o jogo ganha para cada usuário que instalar o jogo. Ela se difere da ARPDAU pois ela calcula a média do quanto cada usuário gasta no total com o jogo e não diariamente como a ARPDAU. Além disso ARPU de um usuário está ligado também a sua aquisição, de onde os usuários estão vindo? Por tanto mesmo que seu ARPU seja bom hoje, pode ser que amanha entrem muitos usuários sendo que nenhum deles está disposto a pagar por algo no jogo, entao ARPU bom não garante sucesso de um jogo.

2.3.2.3.3 - eCPI
Effective Cost Per Install (eCPI) significa custo por instalação, ou seja quanto o desenvolvedor tem que gastar em termos de propaganda ou qualquer outro tipo de aquisição para conseguir um usuário. É importante que esse numero seja menor que seu ARPU pois se o desenvolvedor gastar U$ 1,00 para conseguir um usuário, e seu ARPU for de U$ 0,50, é muito provavel que o desenvolvedor tenha um prejuízo de U$0,50 por usuário que adquirir com propagandas.

2.3.2.3.4 - LTV # rever
Lifetime value (LTV) significa o valor vitalício é uma métrica similar ao ARPU porém seu foco é rever

2.3.2.3.5 - Conversion Rate
Conversion Rate é a razão de conversão, quanto dos seus usuários se convertem para clientes, ou seja usuários pagantes. Na maioria dos games, 1% ou 2% apenas irão fazer qualquer tipo de compra dentro do jogo, em jogos de sucesso esse valor varia de 3 a 6%, e existem jogos que chegam a 10% ou mais porém normalmente esses jogos são jogos para um nicho especifico por tanto sua audiencia costuma ser menor

2.3.2.3.6 - ARPPU
Average Revenue Per Paying User (AR-P-PU) é a média de receita que cada usuário que é pagante gera, essa é uma metrica que varia bastante entre U$ 5,00 e U$ 20,00 pois isso depende também da relação de quantos desses usuários são "Peixinhos" (Usuários que gastam U$ 1,00 por mês), "Golfinhos" (Usuários que gastam entre U$ 5,00 e U$ 10,00 por mês), ou "Baleias" (Usuários que gastam entre U$ 20,00 e U$ 100,00 por mês).



curva de sucesso Josh Michaels
$$$ custos e ganhos
estratégias: propaganda x jogo pago x in-app
