Jogos eletrônicos fascinam pessoas de todas as idades desde 1970. Naquela época, era comum elas se divertirem com clássicos do Atari como por exemplo o jogo Pong. Desde então, os video-games deram um salto em questão de tecnologia: gráficos, inteligência artificial, equipe de desenvolvimento, história, etc. Eles ficaram mais complexos e com isso a expectativa de alta qualidade por parte dos jogadores também aumenta. Para tentar amenizar a alta complexidade exigida dessa geração de jogos, ferramentas de produção são construídas para facilitar o desenvolvimento. Hoje em dia, equipes iniciantes dispõem de "engines" (motores de jogos) que os permitem acelerar o desenvolvimento. Entretanto, muitas vezes o jogo em questão possui necessidades específicas não contempladas pela "engine" que, por ser terceirizada, pode inclusive dificultar a implementação desses requisitos.

Apesar de toda essa dificuldade, a indústria de jogos eletrônicos é uma que move cerca de 60 bilhões de dólares ao ano sendo inclusive maior que Hollywood. Somando o fato de que muitos adolescentes e adultos tiveram suas infâncias marcadas por seus video-games, temos que essa é uma indústria almejada por muitos. Muitas vezes, a paixão por desenvolver jogos é tamanha que mesmo não fazendo parte de uma grande empresa desenvolvedora de jogos, jovens produzem jogos de forma independente. Quando esses veteranos da indústria são perguntados qual a melhor maneira de começar a trabalhar com video-games, quase todas as respostas lhe indicarão a juntar uma equipe motivada e simplesmente começar um projeto. Essa é uma indústria empírica onde o grande valor está na experiência e experimentação.

Sendo uma área que vem ganhando cada vez mais reconhecimento e notoriedade, empreendedores vêem oportunidade em começar sua própria "startup" de video-games. Isso se deve também ao surgimento e popularização de meios de distribuição digital (lojas virtuais, portais de jogos, etc) que facilitam levar o jogo da desenvolvedora aos jogadores. Esse tipo de startup logo ficaria conhecido como desenvolvedor independente, ou apenas "indie". São independentes pois não necessitam mais de uma empresa publicadora para distribuir seus jogos pelas prateleiras de uma loja de varejo ("retail"), eles simplesmente distribuíam virtualmente pela web. Exemplos desses jogos são "Super Meat Boy", "Fez" e "Braid" que estrelam o documentário "Indie Games: The Movie". Esses e outros jogos logo fariam sucesso por trazerem inovações diversas não vistas há muito tempo nos clássicos jogos de console. Assim era lançada uma nova tendência na indústria: os jogos independentes.
