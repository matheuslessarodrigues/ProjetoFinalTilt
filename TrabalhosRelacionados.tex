Jogos eletrônicos fascinam pessoas de todas as idades desde 1970. Naquela época, era comum elas se divertirem com clássicos do Atari como por exemplo o jogo Pong. Desde então, os video-games deram um salto em questão de tecnologia: gráficos, inteligência artificial, equipe de desenvolvimento, história, etc. Eles ficaram mais complexos e com isso a expectativa de alta qualidade por parte dos jogadores também aumenta. Para tentar amenizar a alta complexidade exigida dessa geração de jogos, ferramentas de produção são construídas para facilitar o desenvolvimento. Hoje em dia, equipes iniciantes dispõem de "engines" (motores de jogos) que os permitem acelerar o desenvolvimento. Entretanto, muitas vezes o jogo em questão possui necessidades específicas não contempladas pela "engine" que, por ser terceirizada, pode inclusive dificultar a implementação desses requisitos.

Apesar de toda essa dificuldade, a indústria de jogos eletrônicos é uma que move cerca de 60 bilhões de dólares ao ano sendo inclusive maior que Hollywood. Somando o fato de que muitos adolescentes e adultos tiveram suas infâncias marcadas por seus video-games, temos que essa é uma indústria almejada por muitos. Muitas vezes, a paixão por desenvolver jogos é tamanha que mesmo não fazendo parte de uma grande empresa desenvolvedora de jogos, jovens produzem jogos de forma independente. Quando esses veteranos da indústria são perguntados qual a melhor maneira de começar a trabalhar com video-games, quase todas as respostas lhe indicarão a juntar uma equipe motivada e simplesmente começar um projeto. Essa é uma indústria empírica onde o grande valor está na experiência e experimentação.

Sendo uma área que vem ganhando cada vez mais reconhecimento e notoriedade, empreendedores vêem oportunidade em começar sua própria "startup" de video-games. Isso se deve também ao surgimento e popularização de meios de distribuição digital (lojas virtuais, portais de jogos, etc) que facilitam levar o jogo da desenvolvedora aos jogadores. Esse tipo de startup logo ficaria conhecido como desenvolvedor independente, ou apenas "indie". São independentes pois não necessitam mais de uma empresa publicadora para distribuir seus jogos pelas prateleiras de uma loja de varejo ("retail"), eles simplesmente distribuíam virtualmente pela web. Exemplos desses jogos são "Super Meat Boy", "Fez" e "Braid" que estrelam o documentário "Indie Games: The Movie". Esses e outros jogos logo fariam sucesso por trazerem inovações diversas não vistas há muito tempo nos clássicos jogos de console. Assim era lançada uma nova tendência na indústria: os jogos independentes.

Ainda assim, apesar das novas tendências e facilidades do mercado de ser independente, ser parte de uma das gigantes da indústria ainda é sonho de muitos desenvolvedores. Empresas como Blizzard e Valve entre outras são sonhos de muitos. Elas se destacam por possuirem um ambiente de trabalho descontraido e que estimula a criatividade assim como muitas empresas de entretenimento. Tomando como exemplo a Valve que possui um manual para novos funcionários que quebra muitos paradigmas a respeito do esquema organizacional de uma empresa. Eles prezam pela qualidade de cada membro e vêem a contratação como uma das atividades mais importantes e cruciais para a empresa. Além disso, mantém uma estrutura organizacional totalmente horizontal, ou seja, ninguém é chefe de ninguém.

----

[1] Game Development - Harder Than You Think
(artigo que explica as dificuldades e riscos envolvidos ao se trabalhar com jogos. também traça um comparativo de como era antigamente e atualmente)
http://dl.acm.org/citation.cfm?id=971590

[2] How to get into the games industry – an insiders' guide
(bando de dicas e comentários de veteranos da indústria pra ajudar iniciantes na área)
http://www.theguardian.com/technology/2014/mar/20/how-to-get-into-the-games-industry-an-insiders-guide

[3] Game on: Competition and competitiveness in the video game industry
http://www.academia.edu/766244/Game_on_Competition_and_competitiveness_in_the_video_game_industry

[4] The Crowdfunding Bible
(um estudo aprofundado de como realizar campanhas de crowdfunding. inclui exemplos, análises, dicas e postmortem)
http://www.trafficsurf.com/freepdf/The-Crowdfunding-Bible.pdf

[5] Valve's New Employee Handbook
(manual para novos empregados da valve. mostra como a empresa tem uma organização não ortodoxa e é referência por isso e outras)
http://media.steampowered.com/apps/valve/Valve_NewEmployeeHandbook.pdf

[6] Dustforce Sales Figures
(números e gráficos a respeito dos gastos e receitas do jogo indie Dustforce)
http://hitboxteam.com/dustforce-sales-figures

[7] Indie Game: The Movie
(documentário sobre o desenvolvimento de jogos independentes)
http://www.indiegamethemovie.com/about/
