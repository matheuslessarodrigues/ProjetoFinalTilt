Jogos eletrônicos fascinam pessoas de todas as idades desde 1970. Naquela época, era comum elas se divertirem com clássicos do Atari como por exemplo o jogo Pong. Desde então, os video-games deram um salto em questão de tecnologia: gráficos, inteligência artificial, equipe de desenvolvimento, história, etc. Eles ficaram mais complexos e com isso a expectativa de alta qualidade por parte dos jogadores também aumenta. Para tentar amenizar a alta complexidade exigida dessa geração de jogos, ferramentas de produção são construídas para facilitar o desenvolvimento. Hoje em dia, equipes iniciantes dispõem de "engines" (motores de jogos) que os permitem acelerar o desenvolvimento. Entretanto, muitas vezes o jogo em questão possui necessidades específicas não contempladas pela "engine" que, por ser terceirizada, pode inclusive dificultar a implementação desses requisitos. [1]

Apesar de toda essa dificuldade, a indústria de jogos eletrônicos é uma que move cerca de 60 bilhões de dólares ao ano sendo inclusive maior que Hollywood. Somando o fato de que muitos adolescentes e adultos tiveram suas infâncias marcadas por seus video-games, temos que essa é uma indústria almejada por muitos. Muitas vezes, a paixão por desenvolver jogos é tamanha que mesmo não fazendo parte de uma grande empresa desenvolvedora de jogos, jovens produzem jogos de forma independente. Quando esses veteranos da indústria são perguntados qual a melhor maneira de começar a trabalhar com video-games, quase todas as respostas lhe indicarão a juntar uma equipe motivada e simplesmente começar um projeto. Essa é uma indústria empírica onde o grande valor está na experiência e experimentação. [2][3]

Sendo uma área que vem ganhando cada vez mais reconhecimento e notoriedade, empreendedores vêem oportunidade em começar sua própria "startup" de video-games. Isso se deve também ao surgimento e popularização de meios de distribuição digital (lojas virtuais, portais de jogos, etc) que facilitam levar o jogo da desenvolvedora aos jogadores. Esse tipo de startup logo ficaria conhecido como desenvolvedor independente, ou apenas "indie". São independentes pois não necessitam mais de uma empresa publicadora para distribuir seus jogos pelas prateleiras de uma loja de varejo ("retail"), eles simplesmente distribuíam virtualmente pela web. Exemplos desses jogos são "Super Meat Boy", "Fez" e "Braid" que estrelam o documentário "Indie Games: The Movie". Esses e outros jogos logo fariam sucesso por trazerem inovações diversas não vistas há muito tempo nos clássicos jogos de console. Assim era lançada uma nova tendência na indústria: os jogos independentes. [7]

Uma grande mudança que a cena "indie" trouxe para a indústria foi uma maior colaboração dos membros que a compõem. Desenvolvedores indies não se vêem como competidores. Para que tenham alguma chance de competirem com os grandes títulos das empresas já consolidadas, eles colaboram entre si ajudando uns aos outros com suas experiências no mercado. Uma grande prova disso foram os dados de vendas do jogo "indie" Dustforce disponibilizados pelo estúdio "Hitbox Team" em seu blog [6], tipo de dados que não é facilmente encontrato em um mercado altamente competitivo. Nesse post, constam informações sobre como foi o desempenho de venda do jogo nas plataformas virtuais "Steam" e "Humble Bundle". Além disso, o post também continham os custos necessários para desenvolver o jogo.

Igualmente, o estúdio Vlambeer [8] faz a sua parte para ajudar outros "indies" mundo afora. Criadores do "Super Crate Box" e atualmente desenvolvedores do "Nuclear Throne", costumam contribuir relatando sobre suas experiências na indústria, trabalhando em parceria com outros desenvolvedores, além de também criarem ferramentas gratuitas para facilitar o trabalho de estúdios menores. Eles são uma referência para os desenvolvedores independentes pois são a prova de que é possível criar e manter uma startup de video-game com apenas 2 membros integrantes e sobrevivendo por vários meses com um "burn rate" baixíssimo - à base de miojo.

Possuir meios de se sustentar enquanto uma startup está produzindo seu primeiro projeto é de extrema importância. Sendo assim, desenvolvedores independentes buscam levantar fundos das mais diversas formas, uma delas se destacando por ser muito utilizada: o "crowdfunding". Ele funciona através de uma campanha na qual o estúdio deve engajar seu público alvo a respeito do projeto para que então eles invistam dinheiro em troca de uma recompensa definida pelo estúdio. "Crowdfunding" é especialmente interessante para video-games pois os que investiram no projeto posteriormente irão se tornar jogadores o que permite o jogo ter uma massa crítica de usuários durante o lançamento. Dada a importância e a atenção que essa forma de levantar fundos vem ganhando, surgiram literatura que expõe boas práticas para uma boa campanha assim como projetos financiados com sucesso, como é o caso do livro "The Crowdfunding Bible". [4]

Empreendedores brasileiros também já enxergaram o potencial da indústria. Desde de 2000, a Hoplon [9] já trabalhava com video-games no Brasil e logo viria a desenvolver um dos maiores jogos brasileiros: "Taikodom" [10]. A Hoplon não está sozinha, poucos anos depois iriam surgir outros estúdios influentes na cultura brasileira. A Aquiris [11] ganha espaço com sua excelência técnica com a ferramenta Unity3D [12] como o melhor estúdio que a utiliza na América Latina. Há também a Behold Studios [13], estúdio brasileiro fundado em 2009 e que já foi nomeado e premiado diversas vezes em festivais de jogos independentes. Sem mencionar a Critical Studio que por desenvolver o megalomaníaco Dungeonland [14] pôde provar à indústria local que é possível desenvolver e publicar jogos de grande porte no Brasil. Com todos esses grandes estúdios servindo de exemplo, começam então a surgir novas startups devenvolvedoras que tentam atingir o sonho de criar um negócio sustentável a partir de jogos eletrônicos, como é o caso da BitCake Studio [15] no Rio de Janeiro.

Ainda assim, apesar das novas tendências e facilidades do mercado de ser independente, ser parte de uma das gigantes da indústria ainda é sonho de muitos desenvolvedores. Empresas como Blizzard e Valve entre outras são sonhos de muitos. Elas se destacam por possuirem um ambiente de trabalho descontraido e que estimula a criatividade assim como muitas empresas de entretenimento. Tomando como exemplo a Valve que possui um manual [5] para novos funcionários que quebra muitos paradigmas a respeito do esquema organizacional de uma empresa. Eles prezam pela qualidade de cada membro e vêem a contratação como uma das atividades mais importantes e cruciais para a empresa. Além disso, mantém uma estrutura organizacional totalmente horizontal, ou seja, ninguém é chefe de ninguém.

----

[1] Game Development - Harder Than You Think
(artigo que explica as dificuldades e riscos envolvidos ao se trabalhar com jogos. também traça um comparativo de como era antigamente e atualmente)
http://dl.acm.org/citation.cfm?id=971590

[2] How to get into the games industry – an insiders' guide
(bando de dicas e comentários de veteranos da indústria pra ajudar iniciantes na área)
http://www.theguardian.com/technology/2014/mar/20/how-to-get-into-the-games-industry-an-insiders-guide

[3] Game on: Competition and competitiveness in the video-game industry
http://www.academia.edu/766244/Game_on_Competition_and_competitiveness_in_the_video_game_industry

[4] The Crowdfunding Bible
(um estudo aprofundado de como realizar campanhas de crowdfunding. inclui exemplos, análises, dicas e postmortem)
http://www.trafficsurf.com/freepdf/The-Crowdfunding-Bible.pdf

[5] Valve's New Employee Handbook
(manual para novos empregados da valve. mostra como a empresa tem uma organização não ortodoxa e é referência por isso e outras)
http://media.steampowered.com/apps/valve/Valve_NewEmployeeHandbook.pdf

[6] Dustforce Sales Figures
(números e gráficos a respeito dos gastos e receitas do jogo indie Dustforce)
http://hitboxteam.com/dustforce-sales-figures

[7] Indie Game: The Movie
(documentário sobre o desenvolvimento de jogos independentes)
http://www.indiegamethemovie.com/about/

[8] Vlambeer presskit
(pequeno estúdio que já fez vários jogos independentes de grande repercussão)
http://www.vlambeer.com/press/

[9] Hoplon
(uma das primeiras grandes empresas brasileiras de games)
http://www.hoplon.com/empresa/sobre-a-hoplon/

[10] Taikodom
(um dos maiores jogos brasileiros)
http://taikodom.com.br/

[11] Aquiris
(o melhor estúdio unity3d da américa latina)
http://www.aquiris.com.br/?lang=pt_BR

[12] Unity3D
(motor de jogos muito popular)
http://unity3d.com/

[13] Behold Studios
(estúdio independente brasileiro)
http://beholdstudios.com.br/about/

[14] Critical Studio
(estúdio bastante influente no brasil)
http://www.dungeonlandgame.com/about/

[15] BitCake Studio
(pequeno estúdio independente no Rio de Janeiro)
http://www.bitcakestudio.com/presskit/
